\pagenumbering{roman}
\setcounter{page}{1}

\selecthungarian

%----------------------------------------------------------------------------
% Abstract in Hungarian
%----------------------------------------------------------------------------
\chapter*{Kivonat}\addcontentsline{toc}{chapter}{Kivonat}

% Hungarian abstract to be added if required by institution


\vfill
\selectenglish


%----------------------------------------------------------------------------
% Abstract in English
%----------------------------------------------------------------------------
\chapter*{Abstract}\addcontentsline{toc}{chapter}{Abstract}

\textbf{Background}: Automatic speech recognition (ASR) systems are increasingly deployed in multilingual contexts where the spoken language varies across users and recordings. Two contrasting strategies have emerged: (1) unified multilingual models that handle all languages with a single system, and (2) language-specific specialized models fine-tuned for individual languages. While both approaches have theoretical merits, direct controlled comparisons on identical test conditions are limited in the literature.

\textbf{Objective}: This thesis evaluates two multilingual ASR approaches: (1) LID→ASR (automatic language identification followed by transcription) versus (2) language-hinted ASR (where language is explicitly provided). We compare OpenAI Whisper across both modes, with Wav2Vec2-XLSR-53 for reference, across four languages (Spanish, French, Hungarian, Mongolian) spanning high to low resource levels.

\textbf{Methods}: We conducted 312 controlled experiments using Mozilla Common Voice v11.0, evaluating language identification accuracy, processing efficiency, and cross-language performance. Three Whisper model sizes (39M, 74M, 244M parameters) were tested to assess scaling trade-offs. All experiments were performed on CPU hardware with fully reproducible evaluation scripts.

\textbf{Results}:
\begin{itemize}
\item \textbf{LID Accuracy}: Whisper achieved 99.31\% language identification accuracy across 144 experiments, with only 1 error (Hungarian→Norwegian).
\item \textbf{LID vs Hinted Efficiency}: Surprisingly, LID→ASR was 2.76× faster than language-hinted mode (6.80s vs 18.78s average), contradicting expectations that LID adds overhead.
\item \textbf{Model Scaling}: Processing time scaled 6× from Whisper-tiny (2.28s) to Whisper-small (13.80s), exhibiting sub-linear parameter-to-latency relationship.
\item \textbf{Language Inequality}: Mongolian exhibited dramatic 10--30× slowdown compared to other languages (30.56s vs 2.56--3.27s), with worst-case samples taking 151 seconds---a critical limitation for low-resource languages.
\item \textbf{Multilingual Coverage}: Whisper supported all 4 languages with built-in LID; Wav2Vec2 supported only 2 (Spanish, French), requiring external language detection.
\end{itemize}

\textbf{Conclusions}: LID→ASR is production-ready, achieving 99.31\% accuracy while being faster than language-hinted mode. Whisper's unified multilingual architecture provides significant deployment advantages over language-specific models: smaller footprint (244MB vs 1.2GB for 2 languages), broader coverage (4 vs 2 languages), and built-in language detection. However, severe performance degradation for low-resource languages (Mongolian 10--30× slower) reveals critical inequalities in multilingual AI systems.

\textbf{Contributions}: This thesis provides (1) first systematic evaluation of Whisper's LID capability, (2) discovery that LID improves rather than degrades efficiency, (3) quantification of language inequality in multilingual ASR (10--30× Mongolian slowdown), (4) deployment-focused methodology prioritizing practitioner-relevant metrics, and (5) fully reproducible evaluation framework with open-source code.

\textbf{Keywords}: Automatic Speech Recognition, Multilingual ASR, OpenAI Whisper, Wav2Vec2, Language Identification, Model Scaling, Speech Technology


\vfill
\selectthesislanguage

\newcounter{romanPage}
\setcounter{romanPage}{\value{page}}
\stepcounter{romanPage}